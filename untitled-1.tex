\documentclass{article}
\usepackage[MeX]{polski} 
\usepackage[utf8]{inputenc}  
 \usepackage[final] {graphicx}
 \usepackage{wrapfig}

\title{Tytuł Pracy}
\author{Kajetan Jankowski}

\begin{document}

\maketitle



\section{Wprowadzenie}
\textbf{Ucze} sie systemu LaTeX.



\section{streszczenie pracy}
Jakaś matma , jakiś tekst z internetu, obrazki tabela

\newpage

\part{Wprowadzenie do systemu matematycznego} 
\label{1}
Uczę się systemu \LaTeX. Teraz użyje jakiś symboli matematycznych, które ciężko używa się w Wordzie, tak by było bardzo fajnie i supcio: \\
Słynne równanie:  \\
$$E = mc^2$$ 
jest najpopularniejszym równaniem w fizyce pewnego autora.

jakies matematyczne zapisy

$\left|x+1\right|$


$a+b-c/d=0 \cdot e$

$x^2$

$\sqrt{x+1}$ 
\part{Jakiiś tekst z internetu}
\label{2}
\section{coś o covid}
W ostatnim czasie w mediach, obok epidemiologów, brylują specjaliści od pracy zdalnej. Dziesiątki coachów udziela rad typu: „jak pracować w domu, żeby nie zwariować”. Każdy, kto na dobre zasmakował telepracy, patrzy na takie wskazówki z przymrużeniem oka. Może nie zaszkodzą, ale też niewiele pomogą.

Żeby stać się w pełni efektywnym zdalnym pracownikiem, potrzeba kilku długich miesięcy, w czasie których ćwiczy się sztukę walki z rozpraszaczami, a także unikania obowiązków związanych z codziennym gotowaniem, zakupami dla teściowej lub odbieraniem ubrań z pralni. 
\begin{wrapfigure}{l}{0.5\textwidth}
\centering
\includegraphics[width=0.7\linewidth]{lis.jpg}
\caption{lisek}
\label{fig:wrapfig}
\end{wrapfigure}Nikomu jeszcze nie udało się znaleźć na to cudownej recepty, \textsl{podobnie jak na COVID-19}. Niektórzy, po żmudnym treningu, wchodzą na sam szczyt, inni kapitulują już po kilku dniach. Znajoma opowiadała o koleżance, która wraz z nastaniem stanu epidemiologicznego, otrzymała propozycję przejścia na tryb pracy zdalnej i przyjęła ją z euforią. Świeżo upieczona telepracowniczka przez dwa dni wręcz bombardowała SMS-ami i e-mailami nie tylko swoją przełożoną, ale również dział IT. Trzeciego dnia otrzymała polecenie służbowe nakazujące powrót do biura.
\newpage
\subsection{coś o internecie}
\textbf{Pierwowzór sieci Internet zaczął powstawać w połowie lat 60-tych}. Dowództwo wojskowe Stanów Zjednoczonych potrzebowało szybkiej i niezawodnej sieci, dzięki której mogliby wymieniać informacje podczas wojny. Dodatkowo sieć ta musiała umożliwiać dostarczenie wiadomości nawet w przypadku jej częściej awarii bądź też zniszczenia. Projekt powiódł się, i u schyłku lat 80-tych Internet był już podstawowym sposobem wykorzystywanym przez komputery do przesyłania między sobą danych. Ogromne tempo rozwoju Internetu, w którego skład wchodzą miliony mniejszych sieci komputerowych z całego świata zaczął jednak niepokoić. Chodzi przede wszystkim o to czy przesyłane dane są bezpieczne. Wraz z upowszechnieniem się globalnej sieci pojawił się problem kradzieży poufnych wiadomości (tak zwane włamania elektroniczne). Ludzi zajmujących się tym procederem nazywa się hakerami. Stworzony przez grupę hakerów wirus komputerowy, (czyli szkodliwy program) nazwany przez nich robakiem internetowym, (czyli z ang. Internet Worm) dokonał blokady połączenia ok. sześciu tysiącom serwerów, co stanowiło 10% całej ówczesnej sieci (liczonej wtedy na ok. 60 000 serverów). Po mimo tego liczba komputerów podłączanych do Internetu cały czas rośnie lawinowo.

NSFNET - obrazek  \ref{lis}  wydzielona w połowie lat 80-tych z ARPANETu sieć oparta na 5 bardzo nowoczesnych jak na owe czasy serwerach udostępniona dla uczonych oraz środowisk studenckich, przyczyniła się również do ogromnego wzrostu popularności sieci. Rosnąca popularność sieci sprawiła, że rząd Stanów Zjednoczonych podjął decyzję o udostępnieniu Internetu szerszej liczbie użytkowników i popierał jego rozwój, dzięki czemu na podłączenie do globalnej pajęczyny mogły sobie pozwolić firmy oraz użytkownicy prywatni. Doprowadziło to w 1995 roku do ogromnej liczby podłączonych użytkowników sięgającej 35 000 000 komputerów z około 130 państw z całego świata. Internet bezdyskusyjnie jest siecią informatyczną skupiającą najwięcej komputerów na świecie jednak nie jest to sieć jedyna. Są oczywiście jeszcze inne sieci opierające się na przesyłaniu danych za pomocą łączy telefonicznych. Działanie ich opiera się głównie na transmitowaniu dźwięków, ale udostępniają one tez swoim klientom usługi poczty elektronicznej, połączenie z bazami danych itd. Sieci te nie mogą jednak konkurować z popularnością Internetu, osiągniętą głównie dzięki WWW. Strony internetowe, (czyli z ang. World Wide Web) umożliwiły każdemu użytkownikowi na publikowanie w Internecie nie tylko tekstu, ale również i zdjęć, muzyki gier on-line itd. Sam pomysł stworzenia WWW został opracowany przez Tima Bernresa-Lee ówczesnego pracownika naukowego szwajcarskiego ośrodka badań CERN. Jego idea trafiła na podatny grunt University of Illinois, kierowani przez Marca Adreesena tamtejsi studenci oraz programiści wymyślili i zaprezentowali pierwszą wersję programu umożliwiającego prezentowanie stron internetowych na komputerze użytkownika. Ten pra pra dziadek dzisiejszych przeglądarek internetowych nazywał się NCSA Mosaic. Dzisiejsze najpopularniejsze przeglądarki internetowe to Internet Explorer oraz Opera i Mozilla Firefox. Upowszechnienie przeglądarek Internetowych oraz standardu stron WWW spowodowało kolejny ogromny wzrost popularności Internetu a co za tym idzie podłączeń do niego kolejnych podsieci i kolejnych komputerów. Dzięki sieci zaczęto przesyłać nie tylko tekst, ale też i zdjęcia, filmy, grafiką oraz pliki z muzyką. Do Internetu podłączono muzea i biblioteki, dzięki czemu można przeczytać książkę czy też np. zwiedzić Luwr w nie odchodząc od komputera. Rozwój globalnej sieci trwa nadal, na dzień dzisiejszy szacuje się, że do Internetu podłączonych jest ok. 50 000 000 komputerów z ponad 150 państw, z czego serwery on-line szacuje się na około 10 000 000 komputerów. Internet stał się wszechobecny. Nie sposób wyobrazić sobie dzisiaj żeby w CV czy na wizytówce ktoś nie podał swojego adresu e-mail czy strony WWW a przecież to dopiero początek. Cały czas powiększają się moce obliczeniowe komputerów oraz przepustowości sieci, dlatego też naukowcy uważają, że obecna fala rozwoju globalnej sieci to zaledwie początek. Coraz więcej osób deklaruje chęć podłączenia się do globalnej pajęczyny, i są to już nie tylko wszelkiego typu instytucje naukowe czy uczelniane, ale też firmy prywatne i u użytkownicy indywidualni, którzy chcą korzystać z Internetu prywatnie w domu. Podłączenie do sieci pozwala na zaoszczędzenie mnóstwa czasu i pieniędzy. Wiele firm udostępnia w Internecie nie tylko produkty na sprzedaż, ale również różnego rodzaju porady, wyniki badań, często wręcz instrukcje, co do sposobu postępowania w sprawach finansowych, księgowych, prawnych, serwisy prasowe umieszczają aktualne wiadomości na swoich stronach, prognozy pogody itd. Ponad to każdy z użytkowników Internetu może w każdej chwili sam stać się dostarczycielem informacji, może też sprzedać na aukcjach internetowych towary, reklamować się na swoich stronach WWW itd. Internet to potężne źródło możliwości zwłaszcza dla osób niepełnosprawnych, głuchych, przykutych do wózków inwalidzkich, dla wszystkich ludzi, którzy z różnych przyczyn skazani są na zamknięcie w 4 ścianach.
Sytuacja odnośnie Internetu w naszym kraju nie jest jednak aż tak kolorowa. Co prawda uczelnie oraz szkoły nie mają aż tak wielkich problemów jak jeszcze niedawno (dzięki pomocy Kinu czyli Komitetu Badań Naukowych). Dużo zmieniło się tez dzięki ofertom operatorów telewizji kablowej oraz Telekomunikacji Polskiej (Neostrada) zdrowa konkurencja (a przynajmniej jej namiastka) powoduje, że usługi tanieją i coraz więcej osób może sobie na nie pozwolić. Jednak nadal bardzo często można zaobserwować strach ludzi przed komputerami. Niestey nadal powszechne są sytuacje, gdy zwłaszcza w urzędach na biurkach stoi nowiutki komputer a panie w okienkach używają nadal maszyn do pisania i kalkulatorów a zamiast wysyłania różnych potwierdzeń za pomocą e-maila korzysta się z zawodnych, drogich i bardzo nieterminowych usług Poczty Polskiej. Ponad to wiele osób po prostu nie chce poświęcić kilku dni na efektywne opanowanie prostych przecież zasad obsługi komputera. Zdarza się to zwłaszcza w dużych firmach gdzie kierownikami czy dyrektorami są osoby starszej daty. Zakładają oni stronę WWW firmy oraz adres e-mail, ale tak naprawdę z niego nie korzystają lub zatrudniają specjalnie kogoś, kto robi to za nich. Często jednak firmy tnąc koszty nie mogą sobie pozwolić na zwiększenie zatrudnienia i z tego powodu nie ma osoby odpowiedzialnej za korespondencję elektroniczną, czy uaktualnianie strony WWW a wysyłane do firmy e-maile nie są odczytywane przez nikogo. Bardzo dużo jest jeszcze do zrobienia w Polsce na polu przełamywania niechęci i lęków przed komputeryzacją.









\newpage
\section {obrazek2}
\includegraphics[width=10cm]{lis.jpg} \label{lis}


\section {tabela}
\begin{tabular}{|l|p{4.7cm}|p{4cm}|} \hline
jeden \ref{1} & pierwszy rozdział \\
\hline
dwa \ref{2} & drugi rozdział\\ \hline
\end{tabular}


\begin{thebibliography}{9}

\bibitem{1}
	obrazek lisa: 
  \emph https://www.biolib.cz/IMG/GAL/BIG/317432.jpg
\bibitem{2}
	jakiś tekst z iinternetu o covid: 
\emph https://crn.pl/wywiady-i-felietony/jakis-tekst
\bibitem{3}
	jakiś tekst z internetu o iinternecie: 
\emph https://www.bryk.pl/wypracowania/pozostale/informatyka/13964-krotki-tekst-na-temat-internetu.html

\end{thebibliography}
\end{document}